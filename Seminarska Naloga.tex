\documentclass[a4paper, 12pt]{article}
\usepackage[slovene]{babel}
\usepackage[utf8]{inputenc}
\usepackage[T1]{fontenc}
\usepackage{hyperref}

\hypersetup{
	colorlinks,
	citecolor=black,
	filecolor=black,
	linkcolor=black,
	urlcolor=black
}

\begin{document}
	\begin{titlepage}
		\newcommand{\HRule}{\rule{\linewidth}{0.5mm}}
		\center
		
		%	HEADING SECTIONS
		\textsc{\LARGE Gimnazija Vič}\\[0.5cm]
		\textsc{\Large Tržaška c. 72, 1000 Ljubljana}\\[1.5cm]

		%	TITLE SECTION		
		\HRule \\[0.4cm]
		{ \huge \bfseries Programiranje za mobilne platforme iOS}\\[0.4cm]
		\HRule \\[1.5cm]
		
		%	AUTHOR SECTION	
		\begin{minipage}{0.4\textwidth}
			\begin{flushleft} \large
				\emph{Avtor:}\\
				Vid Drobnič
			\end{flushleft}
		\end{minipage}
		~
		\begin{minipage}{0.4\textwidth}
			\begin{flushright} \large
				\emph{Mentor:} \\
				prof. Klemen Bajec
			\end{flushright}
		\end{minipage}\\[4cm]
		
		%	DATE SECTION
		{\large 13. februar 2017}\\[3cm]
		\vfill
	\end{titlepage}
	
	\section*{Izvleček}
	en izvlecek pac
	
	\section*{Abstract}
	abstract pac.
	
	\pagebreak
	
	\tableofcontents
	\pagebreak
	
	\section{Uvod}
	\paragraph{} Čez celo šolsko leto imamo dijaki veliko nadome"sčanj, ki so dostopna na spletni učilnici. Zaradi nepreglednosti spletne učilnice je pregledovanje nadomeščanj zelo dolgotrajno opravilo. Zato sem se odločil napisati aplikacijo za mobilne naprave z operacijskim sistemom iOS.
	
	\paragraph{} Pred začetkom pisanja aplikacije sem si zadal določene cilje. Poleg prikazovanja urnika z nadomeščanji, sem si kot cilj zastavil tudi prikazovanje jedilnika. Vse zastavljene cilje sem tudi dosegel. Na koncu sem preveril še priljubljenost aplikacije s pomočjo statističnih podatkov, ki sem jih pridobil na Apple-ovi spletni strani za razvijalce.
	
	\section{Operacijski sistem iOS}
\end{document}