\paragraph{}iOS\cite{ios-wiki} je Applov operacijski sistem a mobilne naprave. V osnovi je bil zasnovan za iPhone, danes pa ga uporabljajo tudi iPad, iPod Touch in Apple TV. Prva verzija je bila izdana leta 2007. Zasnovan je na Unix-ovem\cite{unix-wiki} jedru.

\subsection{Programiranje za iOS platformo}
\paragraph{}Apple je za iOS razvil razvijalsko orodje Xcode\cite{xcode-wiki}, ki je na voljo za ra"cunalnike z operacijskim sistemom macOS. Naredili so tudi primerno dokumentacijo\cite{ios-docs} za programiranje za iOS in spletno trgovino imenovane App Store\cite{app_store-wiki}, kamor razvijalci objavimo svoje aplikacije.

\subsubsection{Jeziki}
\paragraph{}"Cez leta se je zamenjalo veliko jezikov za pisanje aplikacij za iOS platformo. Prvotno so se aplikacije pisale v C++\cite{cpp-wiki}. Kasneje ga je zamenjal programski jezik Objective-C\cite{objective-c-wiki}, ki je bil uporabljen samo na Applovih platformah. Leta 2014 je Apple predstavil svoj odprtokodni programski jezik imenovan Swift\cite{swift-wiki}, ki naj bi zamenjal sedaj nekoliko zastarel Objective-C\cite{objective-c-wiki}.

\subsubsection{Zgradba iOS Aplikacije}
\paragraph{}Aplikacije za iOS se shranjujejo v Xcode projektih. To je skupina map in datotek z dolo"ceno strukturo. Prevladujejo Swift oziromva Objective-C (odvisno od tega kater jezik si izberemo za razvoj), .storyboard, .xib in .plist datoteke. V .storyboard in .xib datotekah se nahaja zgradba uporabni"skega vmesnika, v .plist datotekah se nahaja konfiguracija aplikacije (ime aplikacije, ime razvijalca, ...), v Swift oziroma Objective-C datotekah pa se nahaja koda, ki se kasneje izvaja na napravi.	

\paragraph{}Aplikacije je sestvaljena na modelu MVC\cite{mvc-wiki} (Model-View-Controller). Model nam predpi"se tri glavne komponente aplikacije. Komponenta Model skrbi za pridobivanje, shranjevanje in obdelavo podatkov. Komponenta View skrbi za prikazovanje grafi"cnega vmesnika. Komponenta Controller pa povezuje Model in View med seboj. Ko uporabnik pritisne na dolo"ceno del uporabni"skega vmesnika, gre Controller po podatke v Model objekt in nato pove View objektu katere podatke naj prika"ze. 
