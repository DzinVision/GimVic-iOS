\paragraph{}Ko sem aplikacijo dokon"cal, sem jo "se objavil v spletni trgovini App Store. Gre virtualno trgovino po kateri uporabniki iOS pregledujejo razli"cne aplikacije in jih nalo"zijo na svoje naprave 

\paragraph{}Za objavo aplikacije na App Store potrebujemo razvijalski ra"cun za Applove naprave, s katerim se prijavimo v njihov spletni portal. Tam naredimo novo aplikacijo in jim dodelimo unikatno ime, ki ga vidi samo razvijalec (\textit{npr.} com.gimvic.ios). Aplikaciji dodamo tudi slike in opis, lahko pa nalo"zimo tudi promocijski video.

\paragraph{}Pred objavo moramo na razvijalski portal nalo"ziti tudi samo aplikacijo - to je datoteka, ki se dejansko izvaja na napravi. To naredimo preko razvijalskega orodja XCode, v katerem smo aplikacijo tudi napisali. V XCode moramo biti prijavljeni z razvijalskim ra"cunom, v konfiguraciji pa mora imeti aplikacija isto unikatno ime, kot smo ji ga dodelili v spletnem portalu. Nato lahko z gumbom publish aplikacijo nalo"zimo na portal.